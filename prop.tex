%\documentclass[DIV21]{scrartcl}
\documentclass{article}

\usepackage[utf8]{inputenc}
\usepackage[T1]{fontenc}
\usepackage[usenames,dvipsnames]{color}
\usepackage{graphicx}
%\usepackage{longtable}
%\usepackage{float}
%\usepackage{wrapfig}
%\usepackage{soul}
\usepackage{amssymb}
\usepackage{amsmath}
%\usepackage{footnote}
%\usepackage[T1]{fontenc}
%\usepackage{lmodern}
%\usepackage[math]{kurier}
%\usepackage[scaled]{beramono}
%\usepackage{microtype}
%\usepackage{lineno}
%\usepackage[refpage]{nomencl}
%\usepackage{wasysym} %diameter
\usepackage{natbib}
\usepackage[pdftex,a4paper=true,plainpages,bookmarksnumbered,pagebackref,% put after natbib
pdftitle={Light propagation through bent multi-mode fibers},%
pdfauthor={Martin Kielhorn},%
pdfkeywords={optical fibers, mode coupling}, %
pdfsubject={Technical Report}]{hyperref}
%\usepackage{siunitx}
%\usepackage{units}
%\usepackage{listings}
\usepackage{color}
\usepackage{textcomp}
%\usepackage{subfigure}

%\makenomenclature

\newcommand{\vect}[1]{\mathbf{#1}}
\renewcommand{\r}{\vect r}
\renewcommand{\a}{\vect a}
\newcommand{\s}{\vect s}
\newcommand{\vnu}{\mbox{\boldmath{$\nu$}}}
\newcommand{\valpha}{\mbox{\boldmath{$\alpha$}}}
\newcommand{\vbeta}{\mbox{\boldmath{$\beta$}}}
\newcommand{\vmu}{\mbox{\boldmath{$\mu$}}}
\newcommand{\vtau}{\mbox{\boldmath{$\tau$}}}
\newcommand{\vrho}{\mbox{\boldmath{$\rho$}}}
\newcommand{\vvarrho}{\mbox{\boldmath{$\varrho$}}}
\newcommand{\supp}{\mathop{\mathrm{supp}}}
\newcommand{\pois}{\mathop{\mathrm{pois}}}
\newcommand{\step}{\mathop{\mathrm{step}}}
\newcommand{\diag}{\mathop{\mathrm{diag}}}
\def\a{\vect a}
\def\b{\vect b}
\def\k{\vect k}
\def\d{\vect d}
\def\drm{\textrm d}
\def\e{\vect e}
\def\E{\vect E}
\def\f{\vect f}
\def\H{\vect H}
\def\h{\vect h}
\def\c{\vect c}
\def\x{\vect x}
\def\y{\vect y}
\def\z{\vect z}
\def\q{\vect q}
\def\vzeta{\vect \zeta}
\def\p{\vect p}
\def\P{\vect P}
\def\l{\vect l}

\newcommand{\nvect}[1]{\vect{\widehat{#1}}}
%\renewcommand{\i}{\nvect i}
\newcommand{\vi}{\nvect \i}
\def\hc{\nvect c}
\def\hs{\nvect s}
\def\hd{\nvect d}
\def\hx{\nvect x}
\def\hy{\nvect y}

\def\hz{\nvect z}
\def\n{\widehat{\vect n}}
\def\t{\nvect t}
\def\m{\nvect m}
\def\vrho{\boldsymbol\rho}
\def\abs#1{\mathopen| #1 \mathclose|}
\def\({\left(}
\def\){\right)}

\newcommand{\nco}{n_\textrm{co}}
\newcommand{\rco}{r_\textrm{co}}
\newcommand{\neff}{n_\textrm{eff}}
\newcommand{\ncl}{n_\textrm{cl}}
\newcommand{\lmax}{l_\textrm{max}}
\newcommand{\mmax}{m_\textrm{max}}
\newcommand{\umax}{u_\textrm{max}}
\newcommand{\umin}{u_\textrm{min}}
\newcommand{\Ainfty}{A_\infty}

\renewcommand{\O}{\textsf{O}} % oxygen

\begin{document}

Ich fasse hier die Formeln aus 1973marcuse zusammen, die notwendig
sind um Modenkopplung einer gekr\"ummten Faser zu berechnen.

Die Feldverteilung in der Faser kann durch als eine Superposition von
gef\"uhrten Moden, evaneszenten Stahlungsmoden und sich ausbreitenden
Strahlungsmoden ausgedr\"uckt werden:

\begin{align}
\E(x,y,z) = \sum_l\(
\underbrace{\sum_\mu c_{\mu l} \e_{\mu l}(x,y) e^{i\beta_\mu z}}_\textnormal{gef\"uhrt} +
\underbrace{\int_0^{\rco k \ncl} c_l(Q) \e_l(x,y,Q) e^{i\beta(Q)z}\drm Q}_\textnormal{evaneszent} +
\underbrace{\int_{\rco k \ncl}^\infty c_l(Q) \e_l(x,y,Q) e^{i\beta(Q)z}\drm Q}_\textnormal{ausstrahlend}
\)
\end{align}

Hierbei indiziert $l$ die azimuthale Quantenzahl und $\beta_\mu$
beschreibt die Ausbreitungskonstante der gef\"uhrten Mode mit $\mu$
Maxima entlang des Radius. Der Parameter
$Q=\rco\sqrt{\k^2\ncl^2-\beta^2}$ indiziert sowohl evaneszente (mit
imagin\"aren $\beta(Q)$) und sich ausbreitende Strahlungsmoden (mit
reellem $\beta(Q)$).

Meineserachtens kann die Kopplung von gef\"uhrten Moden zu
evaneszenten Strahlungsmoden bei der gebogenen Faser nicht einfach
vernachl\"assigt werden. Marcuse hat die Kopplung mit evaneszenten
Moden in 1973marcuse jedoch nicht diskutiert, deshalb kann ich daf\"ur
erstmal keine Formeln angeben.

Weiterhin unterscheidet Marcuse bei den sich ausbreitenden
Strahlungsmoden zwischen solchen, die sich nahe der Faserachse
ausbreiten ($\beta\approx \ncl k$) und denen, die unter gr\"osserem
Winkel von der Faser abgestrahlt werden ($\beta<\ncl k$). F\"ur
letztere vernachl\"assigt Marcuse den Einfluss der Fasergeometrie auf
das elektromagnetische Feld, so dass er einfach mit zylindrischen
Freistrahlmoden arbeiten kann. Ich vermute, dass f\"uhr unser Problem
der gekr\"ummten Fasern die Verluste haupts\"achlich in
Strahlungsmoden nahe der Faserachse gehen.

F\"ur eine vollst\"andige Beschreibung m\"usste auch die Kopplung in
r\"uckw\"arts laufenden Moden betrachtet werden. Aber auch hier
vermute ich, dass die Verluste f\"ur gekr\"ummte Fasern
vernachl\"assigbar sind.

F\"ur diese Arbeit setzt sich das Feld also wie folgt zusammen:
\begin{align}
\E(x,y,z) = \sum_l\(
\underbrace{\sum_\mu c_{\mu l} \e_{\mu l}(x,y) e^{i\beta_\mu z}}_\textnormal{gef\"uhrt} +
\underbrace{\int_{\rco k \ncl}^{1.1 \rco k \ncl} c_l(Q) \e_l(x,y,Q) e^{i\beta(Q)z}\drm Q}_\textnormal{ausstrahlend, nah an Achse} +
\underbrace{\int_{1.1 \rco k \ncl}^\infty c_l(Q) \e_l(x,y,Q) e^{i\beta(Q)z}\drm Q}_\textnormal{ausstrahlend, fern der Achse}
\)
\end{align}

Die folgenden Differentialgleichungen beschreibt die Entwicklung der
Energieverteilung in den einzelnen lokalen Moden entlang einer
gest\"orten Faser:
\begin{align}
  \frac{\drm c_{(\mu,l_\mu)}}{\drm z} &= 
  \sum_{l_\nu} \sum_\nu  K_{(\mu,l_\mu)(\nu,l_\nu)} c_{(\nu,l_\nu)} e^{i(\beta_\mu-\beta_{(\nu,l_\nu)})z} + 
  \sum_{l_Q} \int_{\rco k \ncl}^\infty  K_{(\mu,l_\mu)(Q,l_Q)} c_{(Q,l_Q)} e^{i(\beta_\mu-\beta_{l_Q}(Q))z} \drm Q
\end{align}
Die Kopplung zwischen zwei Moden mit Indizes $(\mu,l_\mu)$ und
$(\nu,l_\nu)$ wird durch ein \"Uberlappintegral
$K_{(\mu,l_\mu)(\nu,l_\nu)}$ zwischen den Modenfeldern bestimmt. 

An dieser Stelle ist mir nicht klar, wie ich mit den kontinuierlichen
Strahlungsmoden umgehen muss. Im allgemeinen Fall ergibt sich ein
Integro-Differentialgleichungssystem, aber vielleicht kann man
numerisch leicht implementieren, dass alle Energie die in
Strahlungsmoden gekoppelt wurde endg\"ultig verloren ist.


Die St\"orung, die zur gekr\"ummten Faser f\"uhrt ist gem\"ass
Marcuses Notation:
\begin{align}
  r(x,y,z)&=a-f(z)\cos(\phi)\\
  f''(z) &= 1/R
\end{align}
Wobei der Kr\"ummungsradius $R$ die Biegung der Faserachse
beschreibt. Die azimuthale Abhaenigkeit $r(x,y,z)\sim \cos(m\phi)$ mit
$m=1$ limitiert die Wechselwirkung auf gef\"uhrte Moden mit
$l_\nu=l_\mu\pm 1$, bzw.\ $l_Q=l_\mu\pm 1$ f\"ur die Kopplung von
gef\"uhrten Moden in sich ausbreitende Strahlungsmoden.

Die Summen \"uber die azimuthalen Quantenzahlen in den
Differentialgleichungen beschr\"anken sich also auf nur zwei Terme:
\begin{align}
  \label{eq:coupled-modes}
  \frac{\drm c_{(\mu,l_\mu)}}{\drm z} &= 
  \sum_{l_\nu\in\{l_\mu-1,l_\mu+1\}} \sum_\nu  K_{(\mu,l_\mu)(\nu,l_\nu)} c_{(\nu,l_\nu)} e^{i(\beta_\mu-\beta_{(\nu,l_\nu)})z} + 
  \sum_{l_Q\in\{l_\mu-1,l_\mu+1\}} \int_{\rco k \ncl}^\infty  K_{(\mu,l_\mu)(Q,l_Q)} c_{(Q,l_Q)} e^{i(\beta_\mu-\beta_{l_Q}(Q))z} \drm Q
\end{align}

Marcuse separiert die Kopplungskonstante $K$ in das \"Uberlappintegral
$\bar K$ und die Funktion $f(z)$, die die St\"orungsgeometrie
beschreibt:
\begin{align}
  K_{\mu \nu} &= \bar K_{\mu \nu} f(z) 
\end{align}
Nur die Fourierkomponente der St\"orung mit der Frequenz
$\beta_\nu-\beta_\mu$ vermittelt eine Kopplung zwischen den
Moden. Damit l\"asst sich die Kopplungskonstante f\"ur eine gebogene
Faser mit Kr\"ummungsradius $R$ der Faserachse ausdr\"ucken als:
\begin{align}
  K_{\mu \nu} &=  -\frac{1}{(\beta_\nu -\beta_\mu)^2} \frac{1}{R} \bar K_{\mu \nu} 
\end{align}

Marcuse hat das \"Uberlappintegral $\bar K$ analytisch
ausgerechnet. Es gibt folgende F\"alle:



Kopplung zwischen zwei gef\"uhrten Moden:
\begin{align}
  \bar K_{\mu \nu} &= \frac{e_{\mu\nu m}}{\sqrt{e_\mu e_\nu}}
  \frac{u_\nu u_\mu J_\nu(u_\nu) J_\mu(u_\mu)}
       {2 i \rco^3 n k \sqrt{
           |J_{\nu-1}(u_\nu)J_{\nu+1}(u_\nu)
           J_{\mu-1}(u_\mu)J_{\mu+1}(u_\mu)|}}
\end{align}
Der Faktor $e_{\mu\nu m}$ verschwindet ausser wenn $\mu=\nu\pm m$.

Kopplung zwischen gef\"uhrter Mode $\nu$ und sich nahe der Faserachse
ausbreitende Strahlungsmode $\mu$ mit $\beta\approx \ncl k$:
\begin{align}
  \bar K_{\mu \nu} &= \frac{e_{\mu\nu m}}{\sqrt{e_\mu e_\nu}}
  \frac{\sqrt{\frac{\nco}{\ncl}-1}\ u_\nu\sqrt{\color{red}Q}\  J_\nu(u_\nu) {\color{red}J_\mu(u_\mu)}}
       {i \pi \rco^{5/2} \sqrt{|J_{\nu-1}(u_\nu)J_{\nu+1}(u_\nu)|}\ 
         {\color{red}|u_\mu J_{\mu-1}(u_\mu)H_{\mu}^{(1)}(Q)-QJ_\mu(u_\mu)H_{\mu-1}^{(1)}(Q)|}}
\end{align}
Hier habe ich die Terme farblich hervorgehoben \"uber die sich das
Integral der kontinuierlichen Ausbreitungskonstante erstrecken
muss. Dieses Integral muss f\"ur alle betrachteten Moden zur
Bestimmung der \"Uberlappkonstante $\bar K$ einmalig numerisch
berechnet werden. Mit diesen Resultaten k\"onnen dann Simulationen
f\"ur verschiedene Biegeradien ohne nochmalige Berechnung des
Integrals erfolgen.

Kopplung zwischen gef\"uhrter Mode $\nu$ und sich von der Faserachse
entfernende Strahlungsmode $\mu$:
\begin{align}
  \bar K_{\mu \nu} &= \frac{e_{\mu\nu m}}{\sqrt{e_\mu e_\nu}}
  \frac{\sqrt{n\(\frac{\nco}{\ncl}-1\)k\color{red}{\beta Q}}\ u_\nu  J_\nu(u_\nu) \color{red}{J_\mu(Q)}}
       {i \pi \sqrt{2 \rco ({\color{red}\beta}^2+n^2k^2)|J_{\nu-1}(u_\nu)J_{\nu+1}(u_\nu)|}}
\end{align}
In der obigen Formel gilt $n\approx\ncl$.

Obige Ausdr\"ucke enthalten die folgenden Gr\"ossen:
\begin{align}
  e_\nu &= \begin{cases} 2 & \nu=0 \\ 1 & \nu\not=0 \end{cases}
\end{align}

Falls nicht anders spezifiziert verschwindet $e_{\mu\nu m}$.  Falls
die einfallende Mode die azimuthale Abh\"angigkeit $\cos\nu\phi$
aufweist und die St\"orungsmode $\cos\mu\phi$:
\begin{align}
  e_{\mu\nu (m=1)} &= \begin{cases} 
    2 & \nu = 0, \mu = 1\ \textnormal{or}\ \nu = 1, \mu = 0  \\
    1 & 0<\mu = \nu \pm 1\ \textnormal{or}\ 0<\mu=1-\nu 
  \end{cases}
\end{align}
Einfallende Mode:  $\sin\nu\phi$; St\"orungsmode: $\sin\mu\phi$
\begin{align}
  e_{\mu\nu (m=1)} &= \begin{cases} 
    0 & \nu = 0\ \textnormal{or}\ \mu=0 \\ 
    +1 & 0<\mu = \nu \pm 1\\
    -1 & 0<\mu=1-\nu 
  \end{cases}
\end{align}
Einfallende Mode:  $\cos\nu\phi$; St\"orungsmode: $\sin\mu\phi$
\begin{align}
  e_{\mu\nu (m=1)} &= 0
\end{align}



Ich habe die griechischen Buchstaben aus Marcuse in Faserparameter aus
Snyder/Love umgewandelt:
\begin{align}
  \kappa &= \sqrt{\nco^2 k^2-\beta_\nu^2} = U/\rco \\
  \gamma &= \sqrt{\beta^2-\ncl^2 k^2} = W/\rco \\
  \rho &= \sqrt{\ncl^2 k^2-\beta^2} = i\gamma = Q/\rco  \\
  \sigma &= \sqrt{\nco^2 k^2-\beta^2} = U/\rco\quad \textnormal{Strahlungsmode} 
\end{align}

Die Integrale \"uber die Strahlungsmoden in Gleichung
(\ref{eq:coupled-modes}) m\"ussen meineserachtens diskretisiert
werden, so dass verh\"altnism\"a\ss ig viele zus\"atzliche Eintr\"age
zu den Kopplungsgleichungen hinzukommen.

In 2009Huang wird eine bessere Methode vorgestellt, um die Kopplung zu
den Strahlungsmoden zu berechnen. Sie modifizieren die Geometrie der
Faser. Anstelle einer unendlich ausgedehnten Region, f\"uhren sie in
einem gewissen Abstand im Cladding einen 'perfectly matched layer'
ein, gefolgt von einer perfekt reflektierenden Oberfl\"ache ein. Damit
weist das Problem keine Strahlungsmoden mit kontinuierlichen
$\beta-$Spektrum mehr auf, sondern besitzt eine endliche diskrete
Anzahl von Moden mit komplexzahligen $\beta$. Sie zeigen, dass diese
diskreten Strahlungsmoden einfach mit in die Kopplungsgleichung
(\ref{eq:coupled-modes}) eingef\"ugt werden k\"onnen.

\end{document}
