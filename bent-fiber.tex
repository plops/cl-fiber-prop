
%\documentclass[DIV19,twocolumn]{scrartcl}

\documentclass{article}

%\includeonly{spatio-angular}

%\usepackage{enumitem}
%\setitemize{noitemsep,topsep=0pt,parsep=0pt,partopsep=0pt}

%\usepackage{etoolbox}

\pdfminorversion=5
% 9 is best zlib compression i use 0 while writing in the hope that
% git can track changes more easily, set 0 0
\pdfcompresslevel=0
% 3 is maximum for now
\pdfobjcompresslevel=0


\usepackage[utf8]{inputenc}
\usepackage[T1]{fontenc}
\usepackage[usenames,dvipsnames]{color}
%\usepackage[onehalfspacing]{setspace} 
%\usepackage[draft]{graphicx}
\usepackage{graphicx}
%\usepackage{longtable}
%\usepackage{float}
%\usepackage{wrapfig}
%\usepackage{soul}
\usepackage{amssymb}
\usepackage{amsmath}
%\usepackage{examplep}
%\usepackage{pdfcomment}  % breaks hyperref metadata

% choose font here: http://www.tug.dk/FontCatalogue/mathfonts.html
%\usepackage[T1]{fontenc}
%\usepackage[math]{iwona}
%\usepackage[math]{kurier}
%\usepackage{cmbright}
%\usepackage[scaled]{beramono}

%\usepackage{footnote}

%\usepackage{microtype}
%\newcommand\Small{\fontsize{9}{9.2}\selectfont}
%\newcommand*\LSTfont{\Small\ttfamily\SetTracking{encoding=*}{-60}\lsstyle}
%\newcommand*\cmafont{\fontsize{8}{8.2}\selectfont\ttfamily\SetTracking{encoding=*}{-60}\lsstyle}


%\usepackage{esvect} % vector CB in raytrace chapter



%\newcont\pdfcompresslevel


%\usepackage[hypertex,breaklinks]{hyperref} 
% breaklinks only seems to work with dvipdfm,
% otherwise urls have no
% line breaks

%\usepackage[showrefs,showcites,ignoreunlbld]{refcheck} % for draft, uncomment for final
%\usepackage[notref,notcite]{showkeys}

% \usepackage[disable]{todonotes} % for draft, disable for final

%\usepackage{lineno}
%\usepackage[refpage]{nomencl}
%\special{background Black}\special{color Green}
%\usepackage[utf8x]{inputenc} 
%\usepackage[T2A]{fontenc} % for the russian reference
%\usepackage{wasysym} %diameter
% http://www.andy-roberts.net/misc/latex/latextutorial3.html

%\usepackage{url} % natbib.pdf p.11 break urls up, seems to be done
                 % with hyperref, though

\usepackage{natbib}

\usepackage[pdftex,a4paper=true,plainpages,bookmarksnumbered,pagebackref,% put after natbib
pdftitle={Light propagation through bent multi-mode fibers},%
pdfauthor={Martin Kielhorn},%
pdfkeywords={optical fibers, mode coupling}, %
pdfsubject={Technical Report}]{hyperref}


%\usepackage{siunitx} %sudo apt-get install texlive-science
\usepackage{units}


% for app_hilo
%\usepackage{listings}
\usepackage{color}
\usepackage{textcomp}
%\usepackage{subfigure}


% \listfiles % show which files are loaded by tex

\bibpunct{(}{)}{;}{a}{}{,}
%\makenomenclature
\newcommand{\vect}[1]{\mathbf{#1}}
\renewcommand{\r}{\vect r}
\renewcommand{\a}{\vect a}
\newcommand{\s}{\vect s}
\newcommand{\vnu}{\mbox{\boldmath{$\nu$}}}
\newcommand{\valpha}{\mbox{\boldmath{$\alpha$}}}
\newcommand{\vbeta}{\mbox{\boldmath{$\beta$}}}
\newcommand{\vmu}{\mbox{\boldmath{$\mu$}}}
\newcommand{\vtau}{\mbox{\boldmath{$\tau$}}}
\newcommand{\vrho}{\mbox{\boldmath{$\rho$}}}
\newcommand{\vvarrho}{\mbox{\boldmath{$\varrho$}}}
\newcommand{\supp}{\mathop{\mathrm{supp}}}
\newcommand{\pois}{\mathop{\mathrm{pois}}}
\newcommand{\step}{\mathop{\mathrm{step}}}
\newcommand{\diag}{\mathop{\mathrm{diag}}}
\def\a{\vect a}
\def\b{\vect b}
\def\k{\vect k}
\def\d{\vect d}
\def\e{\vect e}
\def\E{\vect E}
\def\f{\vect f}
\def\H{\vect H}
\def\h{\vect h}
\def\c{\vect c}
\def\x{\vect x}
\def\y{\vect y}
\def\z{\vect z}
\def\q{\vect q}
\def\vzeta{\vect \zeta}
\def\p{\vect p}
\def\P{\vect P}
\def\l{\vect l}

\newcommand{\nvect}[1]{\vect{\widehat{#1}}}
%\renewcommand{\i}{\nvect i}
\newcommand{\vi}{\nvect \i}
\def\hc{\nvect c}
\def\hs{\nvect s}
\def\hd{\nvect d}
\def\hx{\nvect x}
\def\hy{\nvect y}

\def\hz{\nvect z}
\def\n{\widehat{\vect n}}
\def\t{\nvect t}
\def\m{\nvect m}
\def\vrho{\boldsymbol\rho}
\def\abs#1{\mathopen| #1 \mathclose|}
\def\({\left(}
\def\){\right)}

\newcommand{\nco}{n_\textrm{co}}
\newcommand{\rco}{r_\textrm{co}}
\newcommand{\neff}{n_\textrm{eff}}
\newcommand{\ncl}{n_\textrm{cl}}
\newcommand{\lmax}{l_\textrm{max}}
\newcommand{\mmax}{m_\textrm{max}}
\newcommand{\umax}{u_\textrm{max}}
\newcommand{\umin}{u_\textrm{min}}
\newcommand{\Ainfty}{A_\infty}

\renewcommand{\O}{\textsf{O}} % oxygen

% conclusions of paragraphs in the margin
%\usepackage[marginparwidth=2.5cm]{geometry}
\setlength{\marginparwidth}{2.2\marginparwidth}
\reversemarginpar
\newcommand{\cma}[1]{\marginpar{\cmafont{#1}}}
% include eps or pdf file, that was generated by inkscape, depending
% on if pdflatex or latex processes this file. latex allows a faster
% development cycle but pdflatex generates a smaller and better final
% pdf output
\def\svgending{\ifx\pdfoutput\undefined% 
  .eps_tex% 
  \else%
  .pdf_tex%
  \fi}

% use \svginput{1}{bla} to include bla.svg, make sure you keep this in
% one line, so that make can automatically find the dependencies with
% sed
\newcommand{\svginput}[2]{{\def\svgscale{#1}\input{#2\svgending}}}

\def\pdfending{\ifx\pdfoutput\undefined% 
  _vector.eps% 
  \else%
  _vector%
  \fi}
\newcommand{\pdfinput}[2]{\includegraphics[width=#1]{#2\pdfending}}

% example call \imagw{8cm}{bla.jpg}{bla}{caption abl}
\newcommand{\imagw}[4]{
  \begin{figure}[!hbt]
    \centering
    \includegraphics[width=#1]{#2}
    \caption{#4}
    \label{fig:#3}
  \end{figure}
}

\def\jpgending{\ifx\pdfoutput\undefined% 
  .eps% 
  \else%
  %
  \fi}
% use like this \jpginput{8cm}{imagefile}{caption ...  make sure all
% characters until the opening brace for the caption are on one line
\newcommand{\jpginput}[3]{\imagw{#1}{#2\jpgending}{#2}{#3}}

% this is for plots that are generated by gnuplot
\newcommand{\gnuplotinput}[2]{\begin{figure}[!hbt]%
    \centering%
    \includegraphics{#1_gnuplot}%
    \caption{#2}%
    \label{fig:#1}%
  \end{figure}}

\newcommand{\celegans}{\emph{C.~elegans}}
\DeclareMathOperator{\sign}{sign}
\DeclareMathOperator*{\sinc}{sinc}
\DeclareMathOperator*{\rect}{rect}

% reference to picture
\newcommand{\figref}[1]{\figurename~\ref{#1}}
\title{Light propagation through bent multi-mode fibers} % i don't call make title
\author{Martin Kielhorn}
% short summary at the beginning of a section
\newenvironment{summary}{\begin{quote}\small}{\end{quote}}

\begin{document}
\maketitle
\section{Introduction}
\section{Modes of a multi-mode fiber}
Das elektrische Feld einer rotationssymmetrischen Faser kann dargestellt werden als
\begin{align}
\E(r,\phi,z,t)=\P a \psi(r) e^{i(\omega t-\beta z \pm l \phi)}
\end{align}
mit einem beliebigen konstanten Polarisationsvektor $\P$, der
Ausbreitungskonstante $\beta$, der Amplitude $a$ und der
Drehimpulsquantenzahl $l$.

Das skalare Feld $\psi(r)$ ergibt sich als L\"osung der Differentialgleichung
\begin{align}
\label{eq:bessel-dgl}
\(\bullet_{,r^2} + \bullet_{,r}/r +k_0^2 n^2(r) -\beta^2 -l^2/r^2\)\psi(r) = 0
\end{align}
Ich beschr\"anke mich hier auf Stufenindexfasern mit kleinem
Indexkontrast. Ihre Brechzahlverteilung ist definiert als
\begin{align}
n(r)&=\nco\sqrt{1-2\Delta\theta(r-\rco)} \\
\theta(r)&=\begin{cases}0 &r<0,\\ 1 & r\ge 0\end{cases} \\
\Delta &= \frac{\nco^2-\ncl^2}{2\nco^2} \approx \frac{\nco-\ncl}{\nco}
\end{align}
mit der Stufenfunktion $\Delta(r)$ und dem Profilh\"ohenparameter
$\Delta$. An der Sprungstelle des Brechungsindex m\"ussen sowohl
$\psi$ als auch $\psi_{,r}$ stetig \"ubergehen. F\"ur st\"uckweise konstante Brechzahl $n(r)$ ist Gleichung \eqref{eq:bessel-dgl} die Besselsche Differentialgleichung und hat die L\"osung
\begin{align}
  R&=r/\rco\\
  \psi(R)&=\frac{1}{\sqrt{N}}\begin{cases}
  J_l(uR)/J_l(u), & 0\le R\le 1\\
  K_l(wR)/K_l(w), & 1<R<\infty 
  \end{cases}\label{eq:psi}\\
  u &= \rco k_0 \sqrt{\nco^2-\neff^2} \\
  w &= \rco k_0 \sqrt{\neff^2-\ncl^2} 
\end{align}

[Snyder 13-9a] zeigt, dass man in zylindersymmetrischen Fasern zwei
senkrecht zueinander polarisierte Moden hat, mit gleicher
Ausbreitungskonstante:

\begin{align}
  E_x &= \psi_0(r) \exp(i\beta z)\quad H_y = \sqrt{\frac{\epsilon_0}{\mu_0}} \nco E_x\\
  E_y &= \psi_0(r) \exp(i\beta z)\quad H_x = -\sqrt\frac{\epsilon_0}{\mu_0} \nco E_y
\end{align}

Mit Permittivit\"at des Vakuums ist definiert als $\epsilon_0 =
\mu_0^{-1}c^{-2} \approx \unit[8.854 187 817 \times 10^{-12}]{F/m}$, wobei 
$c=\unit[299792458]{m/s}$ und
$\mu_0=
\unit[4\pi\times10^{-7}]{H/m}$.

Man nennt die Moden LP-moden um auszudr\"ucken, dass die Felder
\"uberall in die gleiche Richtung polarisiert sind. F\"ur $l>0$ gibt
es zwei weitere Moden mit gleicher Ausbreitungskonstante, also
insgesamt vier. In h\"oherer Ordnung unterscheiden sich die
Ausbreitungskonstanten dieser Moden [Snyder p. 304].

Die Terme im radialen Feld in Gleichung \eqref{eq:psi} sind so
skaliert, dass $\psi$ auf dem Rand $r=\rco$ stetig ist. Die Bedingung,
dass auch die Ableitung $\psi_{,r}$ am \"Ubergang vom Kern zum Mantel
stetig ist liefert die Eigenwertgleichung, oder charakteristische
Gleichung, zur Bestimmung des Modenparameters $u$
\begin{align}
  u\frac{J_{l+1}(u)}{J_l(u)}&=w\frac{K_{l+1}(w)}{K_l(w)}\\
  v^2 &= u^2+w^2\\
  v &= \rco k_0 \sqrt{\nco^2-\ncl^2}
\end{align}

F\"ur gro\ss e $l$ verhinderten roundoff Fehler die numerische
Auswertung. Deshalb habe ich die rechte Seite folgendermassen
approximiert (f\"ur $w<0.1 \sqrt{1+l}$):
\begin{align}
  w\frac{K_{l+1}(w)}{K_l(w)} \sim \begin{cases}
    -\frac{1}{\log(w/2)+\gamma} &l=0\\
    2l &\textrm{sonst}
    \end{cases}
\end{align}
mit der Eulerzahl $\gamma=0.577215664901532860$. Aber f\"ur $l=0$,
brauche ich diese N\"aherung nie anwenden.

\begin{figure}[!hbt]
  \centering
  \includegraphics{plots/char-v30-l10.pdf}
  \caption{Verlauf der charakteristischen Funktion f\"ur $v=30$ (rot)
    und eine Darstellung der Nullstellen der Besselfunktion als
    pr\"agnante Peaks in $\log|J_{10}(u)|$ (gr\"un). Die horizontale
    Achse ist $u$.}
  \label{fig:char}
\end{figure}

Das n\"achste Problem ergibt f\"ur die rechte Seite bei $l\approx 149$
und $u=1$. Abhilfe schafft die folgende N\"aherung f\"ur gro\ss e $l$
[Stegun 9.3.1]:
\begin{align}
  J_\nu(z)&\sim \frac{1}{2\pi\nu}\(\frac{ez}{2\nu}\)^\nu \\
  u\frac{J_{l+1}(u)}{J_l(u)} &\sim \frac{u^2}{2l} + O(l^{-2})
\end{align}

% j(l,z):=1/sqrt(2*%pi*l)*(%e*z/(2*l))^l; taylor(expand(z*j(l+1,z)/j(l,z)),l,inf,2);

Man kann zeigen, dass diese Gleichung f\"ur $0<v<2.405$ nur eine
einzig L\"osung hat (single-mode). Der Wertebereich f\"ur m\"ogliche
L\"osungen fuer $u$ liegt zwischen 0 und $v$. Bemerkenswert ist, dass
die durch $J_l(u)$ hervorgerufenen Polstellen in der
charakteristischen Gleichung jeweils eine Mode abgrenzen\footnote{Habe
  ich aus Quellcode von H.~Gross erfahren.}. Die Nullstellen der
Besselfunktion k\"onnen leicht mit einer Funktion aus einer
numerischen Bibliothek [Temme1979] bestimmt werden und die Eigenwerte
$u$ k\"onnen in den damit ermittelten Intervallen mit einer
Nullstellensuche, z.B. Brents Methode [Press2007], berechnet werden.
Bei der aktuellen Implementierung funktioniert diese Modensuche bis
ungef\"ahr 7000 Moden. Bei gr\"osseren $v$-Parametern kommt es zu
floating point overflows bei der Brechnung von $K_{l+1}(w)$ f\"ur etwa
$l>100$ und $w\approx 4$, das Problem liegt also bei zu gro\ss em
$l$. Zur Zeit ist mir nicht klar, ob ein anderer Algorithmus zur
Berechnung der charaketeristischen Funktion ausreicht oder ob zu einem
anderen Datentyp (insbesondere 128-bit quadfloat) gewechselt werden
muss. Das Verh\"altnis von Besselfunktionen kann man als continued
fraction schreiben [Stegun 9.1.73]. 

In [1964Marcatili] wird eine N\"aherungsloesung fuer die Eigenwerte
eines hohlen Wellenleiters entwickelt f\"ur Wellenleiterradien viel
gr\"o\ss er als die Wellenl\"ange $ka = 2\pi a/\lambda \gg |\nu|
u_{nm}$ (mit komplexem Index $\nu$). Uebertragen auf unser
Eigenwertproblem ergibt sich:

% k(n,z) := sqrt(%pi/(2*z))*exp(-z)*(1+(4*n*n-1)/(8*z)); taylor(k(n+1,z)/k(n,z),z,inf,1);
% k(n,z) := (n-1)!/2 *(2/z)^n;taylor(k(n+1,z)/k(n,z),z,0,1);
\begin{align}
  \lim_{z\rightarrow \infty} K_\nu(z) &\sim \sqrt{\frac{\pi}{2z}} e^{-z}\(1+\frac{\mu-1}{8z}\), \quad \mu=4\nu^2 \\ % stegun 9.7.2
  \lim_{z\rightarrow \infty} \frac{K_{\nu+1}(z)}{K_{\nu}(z)} &\sim 1+\frac{2\nu+1}{2z} \\
  \lim_{z\rightarrow 0} K_\nu(z) &\sim \frac{(\nu-1)!}{2} \(\frac{2}{z}\)^\nu \\ % snyder 37-86
  \lim_{z\rightarrow 0     } \frac{K_{\nu+1}(z)}{K_{\nu}(z)} &\sim \frac{2\nu}{z}
\end{align}
Der Fall $z\rightarrow \infty$ macht hier keinen Sinn, denn $w$ kann
maximal $v$ werden.
Auf der anderen Seite, bei gut gef\"uhrten Moden mit kleinem $w$ gilt:
\begin{align}
  u\frac{J_{l+1}(u)}{J_l(u)}&=2l \\
  \lim_{z\rightarrow 0} J_\nu(z) &= (z/2)^\nu/\nu!  \\
    \lim_{z\rightarrow 0} \frac{J_{\nu+1}(z)}{J_{\nu}(z)} &\sim \frac{z}{2(\nu+1)} \\
    \frac{u^2}{l+1} &= l
\end{align}
Ich verstehe nicht wie man da perturbation technique anwenden soll.


In [Snyder 14-7 und 12-3] wird beschrieben wie man eine
N\"aherungsl\"osung f\"ur $u$ bei gro\ss en $v$ und $l\ge 1$ bekommt
[[8] ist Kogelnik 1975 integrated optics p 53-61].

\begin{align}
  \lim_{v\rightarrow\infty }u \rightarrow u_\infty \(1-\frac{2l}{v}\)^\frac{1}{2l}
\end{align}

Die Angaben f\"ur $u_\infty$ werden im Buch nur f\"ur nicht-LP Moden gemacht:

\begin{align}
  J_1(u_\infty) = 0  &\quad \textrm{TE}_{0m}, \textrm{TM}_{0m}\\
  J_0(u_\infty) = 0  &\quad \textrm{HE}_{1m}\\
J_{l+1}(u_\infty) = 0  &\quad \textrm{EH}_{lm}\\
J_{l-1}(u_\infty) = 0  &\quad \textrm{HE}_{lm}, l> 1
\end{align}

Mit Hilfe der WKB N\"aherung k\"onnen einfachere Formeln f\"ur die
h\"oheren Moden (bei $v,u,l\gg 1$ hergeleitet werden:

Die Eigenwertgleichung wird unter der Bedingung $u-l\gg l^{1/3}$:
\begin{align}
  \tan\chi &= \sqrt{\frac{w^2+l^2}{u^2+l^2}}\\
  \chi &= \sqrt{u^2-l^2}-l\arccos(l/u)-\pi/4\\
  \frac{K_{l-1}(w)K_{l+1}(w)}{K_l^2(w)} &\sim \begin{cases}
    \frac{u^2 \sin^2\chi - l^2}{w^2 \cos^2\chi}&  u-l\gg l^{1/3}\\
   1+\frac{1}{\sqrt{w^2+l^2}} & l\gg 1
\end{cases}
\end{align}
Die anderen beiden Ausdr\"ucke k\"onnen zur Orthonomierung mit $N$
benutzt werden.

\begin{align}
  \frac{J_l(z)}{J_{l-1}(z)} &= \frac{1}{2lz^{-1}-} \frac{1}{2(l+1)z^{-1}-} \frac{1}{2(l+2)z^{-1} -}\cdots \\
  &= \frac{\frac{1}{2}z/l}{1-} \frac{\frac{1}{4}z^2/(l(l+1))}{1-} \frac{\frac{1}{4}z^2/(l(l+1)(l+2))}{1-} \cdots
\end{align}


Bei gegebenen Modenparameter $u$ kann die Ausbreitungskonstante
$\beta$ errechnet werden
\begin{align}
\beta = \frac{1}{\rco} \sqrt{\frac{v^2}{2\Delta} - u^2}
\end{align}
Die Normierungskonstante $N$ wird so gew\"ahlt, dass die Intensit\"at
$|a|^2$ wird
\begin{align}
  I &= \frac{1}{2}\left|\int_{A_\infty}\!\!\!\!\! \(\E\times\H^*\)_z  \textrm{d}A \right| = |a|^2
\end{align}
Da der Indexkontrast klein ist ($\Delta \ll 1$), k\"onnen longitudinale Feldkomponenten vernachl\"assigt werden und es gilt
\begin{align}
  \H(r,\phi,z,t) &= \frac{1}{\mu_0 \omega} \k \times \E, \quad\textrm{mit}\ \k = \beta \e_z
\end{align}
Hier ist $\omega=2\pi c/\lambda=k_0c$.
Damit ist die Normierungskonstante 
\begin{align}
  N &= \frac{\pi \rco^2 \nco}{2} \sqrt{\frac{\epsilon_0}{\mu_0}} \frac{v^2}{u^2} \frac{K_{l-1}(w)K_{l+1}(w)}{K_l^2(w)}\\
  \lim_{v\rightarrow\infty} N &= \frac{\pi \rco^2 \nco}{2} \sqrt{\frac{\epsilon_0}{\mu_0}} \frac{v^2}{u^2} \frac{w+1}{w}
\end{align}

Die N\"aherungsgleichung f\"ur grosse $v$ ist aus [Snyder p 318]. Im
Buch wird nicht erl\"autert, wie obige Gleichung hergeleitet werden
kann. Deshalb schreibe ich die Integrale f\"ur Intensitaet in Kern und
Mantel explizit aus.

In meinem Computerprogramm verwende ich nicht den Faktor $\rco^2 \nco
\sqrt{\epsilon_0/\mu_0}$. Das verringert die Anzahl an Argumenten, die
zwischen den Funktionen \"ubergeben werden m\"ussen.

[Snyder p. 264] gibt Felder f\"ur Stufenprofil mit beliebigem
Indexkontrast an. Dann gibt es aber drei verschiedene
Eigenwertgleichungen.

Die Integrale in $N$ sind [Snyder p. 716]: 
% taylor(z^2/2 * (bessel_k(l,a*z)^2- bessel_k(l-1,a*z)*bessel_k(l+1,a*z)),z,inf);
%bk(nu,z):=sqrt(%pi/(2*z))*exp(-z)*(1+(4*nu^2-1)/(8*z));
%assume(a>0);
%taylor(factor(expand(z^2/2 * (bk(l,a*z)^2- bk(l-1,a*z)*bk(l+1,a*z)))),z,inf,3);
\begin{align}
  I_\textrm{core}(z) &= \int z J_m^2(az) \textrm{d} z \\&= \frac{z^2}{2}(J_m^2(az)-J_{m-1}(az)J_{m+1}(az))\\
  I_\textrm{core} {}^1_0 &= \frac{1}{2}(J_m^2(a)-J_{m-1}(a)J_{m+1}(a)) \\
  I_\textrm{clad}(z) &= \int z K_m^2(az) \textrm{d} z \\&= \frac{z^2}{2}(K_m^2(az)-K_{m-1}(az)K_{m+1}(az)) \\
  \lim_{z\rightarrow \infty} K_\nu(z) &\sim \sqrt{\frac{\pi}{2z}} e^{-z}\(1+\frac{\mu-1}{8z}\), \quad \mu=4\nu^2 \\ % stegun 9.7.2
  \lim_{z\rightarrow \infty} I_\textrm{clad}(z)  & = \lim_{z\rightarrow \infty} -\frac{\pi}{4a^2}e^{-2az} \rightarrow 0 \\
  I_\textrm{clad} {}^\infty_1 &= -\frac{1}{2}(K_m^2(a)-K_{m-1}(a)K_{m+1}(a)) 
\end{align}


\begin{figure}[hbtp]
  \centering
  \svginput{1}{modes}
  \caption{Modenfelder f\"ur $v=30$ in einer Matrix mit radialer
    Modenzahl $m$ und Drehimpulsquantenzahl $l$. Weiterhin ist auch
    ein Index eingetragen, der alle Moden ausgehend von der Grundmode
    linear indiziert.}
  \label{fig:fields}
\end{figure}
\section{Kubische Interpolation}
Geben sei eine Tabelle von Funktionswerten $y_i(x_i)$
$i=0,\ldots,N-1$. Die lineare Interpolation in einem Interval
$[x_j,x_{j+1}]$ ist
\begin{align}
  y &= Ay_j + By_{j+1} \\
  A &= \frac{x_{j+1}-x}{x_{j+1}-x_j}\quad B=1-A
\end{align}
Jetzt nehmen wir an, dass wir zus\"atzlich eine Tabelle der zweiten
Ableitungen $y''_i$ haben. Dann k\"onnen wir ein kubische Polynom zu
obigem Ausdruck addieren, dessen zweite Ableitung linear von $y''_j$
auf der linken Seite zu $y''_{j+1}$ auf der rechten Seite variiert.
\begin{align}
  y &= Ay_j + By_{j+1} + Cy''_j + Dy''_{j+1} \\
  C &= (A^3-A)(x_{j+1}-x_j)^2/6 \\
  D &= (B^3-B)(x_{j+1}-x_j)^2/6 
\end{align}
\section{Approximation von sinf}
Eine gut Approximation von f\"ur single float Sinus ist
\begin{align}
  \sin(x)&\approx 0.9999966 x -0.16664815 x^3 \nonumber\\
  &+0.008306204 x^5 -1.8360789\cdot 10^{-4} x^7 
\end{align}
Im Bereich $[-\pi/2,\pi/2]$ ist der Fehler $7.1\cdot 10^{-7}$.
F\"ur die Auswertung berechne ich:
\begin{align}
  \sin(x) \approx (c_1 + c_3 x) x^2 + ( (c_5 +c_7 x)x^2) x^4
\end{align}


%20 7 7.112727e-7 4 4 4 0 | 0.0 0.9999966 0.0 -0.16664815 0.0 0.008306204 0.0 -1.8360789e-4 | 0 16777159/16777216 0 -698973/4194304 0 8918719/1073741824 0 -6308719/34359738368 | sin-F4669396D2C9F733C95387153E86BA05
% from https://github.com/pkhuong/polynomial-approximation-catalogue
\section{Vorgehensweise zur Berechnung der Modenfelder}
Um die Feldverteilung der einzelnen Moden auszurechnen, kann man
einige Zwischenergebnisse wiederverwenden.

Eine N\"aherungsgleichung f\"ur die Anzahl der Moden ist in [Snyder
p257] f\"ur $v\gg 1$ angegeben:
\begin{align}
  M_\textrm{bm} = \textrm{ceiling}\(v^2/2\)
\end{align}

Folgende Funktionen und ihre Ableitungen kann man tabellieren und
sp\"ater mit Spline-Interpolation auswerten um 2D Felder
radialsymmetrisch zu f\"ullen: $\cos(l\phi)$ und $\sin(l\phi)$ mit
$\phi\in[0,2\pi)$ und $l\in[0,\lmax]$. Die maximale Anzahl
Oszillationen auf dem Kernrand h\"angt von $\lmax$ ab. Dies definiert
eine Grenze f\"ur die Pixelgr\"osse des Feldes. Es reicht eine
zyklische lookup Tabelle (oder Interpolation) f\"ur
$[\sin(x),-\cos(x),\sin(x)]$ und $[\cos(x),\sin(x),-\cos(x)]$ f\"ur $l=1$.
[Press2007, Function approximization Ch 5]

% factor(diff(bessel_j(l,x),x,2));
% diff(bessel_j(0,x),x,2);
% factor(diff(bessel_j(1,x),x,2));

Der maximale Wert f\"ur $u$ ist $v$, also reicht f\"ur die
Feldberechnung im Kern $J_l(x)$ mit
$x\in[0,v]$ und $l\in[0,\lmax]$. 
\begin{align}
  J''_0(x) &= \frac{1}{2}\(J_2(x)-J_0(x)\)\\
  J''_1(x) &= \frac{1}{4}\(J_3(x)-3J_1(x)\)\\
  J''_l(x) &= \frac{1}{4}\(J_{l+2}(x)-2J_l(x)+J_{l-2}(x)\)
\end{align}
Hierbei wurde die Eigenschaft $J_{-n}(x) = (-1)^n J_n(x)$ von
Besselfunktionen mit ganzzahliger Ordnung angewendet.

Die Anzahl der radialen Oszillationen ist maximal f\"ur $J_0(u_{0
  m_\textrm{max}} r)$.  Es gilt $u_{0 m_\textrm{max}} \approx v$. Uns
interessiert der Abstand zwischen den zwei a\"ussersten Ringen im
Kernfeld, also $j_{0m_\textrm{max}} - j_{0 (m_\textrm{max}-1)}$, wobei
$j_{nm}$ die $m-$te Nullstelle der Besselfunktion $n-$ter Ordnung
beschreibt.  Dies liefert eine weitere Grenze f\"ur die Pixelgr\"osse
des Feldes.

Die Argumente f\"ur die Funktion $K_l(x)$ erstrecken sich von $w$ nach
unendlich. Das kleinste $w$ ist $\sqrt{v^2-\umax^2}$. Dieser Wert kann
meineserachtens null werden. Verglichen mit $J_l$ sieht $K_l$ sehr
langweilig aus. M\"oglicherweise ist es hilfreich, $K_l(x)\exp(x)$ zu
tabellieren aber die Feldstaerke gef\"uhrter Moden in von uns
betrachteten Multi-mode Fasern faellt ausserhalb des Kernradius
schnell ab und der Einfluss auf Kopplungskonstanten ist
vernachlaessigbar.

% factor(diff(bessel_k(0,x),x,2));
% factor(diff(bessel_k(1,x),x,2));
% factor(diff(bessel_k(l,x),x,2));
% factor(diff(bessel_k(l,x)*exp(x),x,2));

\begin{align}
  K''_0(x) &= \frac{1}{2}\(K_2(x)+K_0(x)\)\\
  K''_1(x) &= \frac{1}{4}\(K_3(x)+3K_1(x)\)\\
  K''_l(x) &= \frac{1}{4}\(K_{l+2}(x)+2K_l(x)+K_{l-2}(x)\)
\end{align}
Hier gilt $K_{-l}(x) = K_l(x)$ [Stegun p 375]. 

F\"ur die skalierte Besselfunktion $k_l(x)=K_l(x)e^x$ sind die zweiten Ableitungen
\begin{align}
  k''_0(x) &= \frac{1}{2}\(k_2(x)-4k_1(x)+3k_0(x)\)\\
  k''_1(x) &= \frac{1}{4}\(k_3(x)-4k_2(x)+7k_1(x)-4k_0(x)\)\\
  k''_l(x) &= \frac{1}{4}\big(k_{l+2}(x)-4k_{l+1}(x)+6k_l(x) \nonumber \\ 
&  -4k_{l-1}(x)+k_{l-2}(x)\big)
\end{align}


Um die Eigenwerte als lineare Sequenz aufzufassen benutzt H.~Gross das
Indexschema, dass durch die roten Zahlen im Mosaik der Modenfelder
angedeutet ist. F\"ur jedes $l$ wird die Anzahl der radialen Moden
$\mmax(l)$ gespeichert und der laufende Index $j$ kann erzeugt werden
durch
\begin{align}
  j &= \begin{cases}
    m & l=0\\
    m+\mmax(0) & l=1\\
    m+\mmax(0)+\mmax(1) & l=1\\
    m+\mmax(0)+ 2\sum_{k=1}^l \mmax(k) & l>1 \\
    m+\mmax(0)+ 2\(\sum_{k=1}^{|l|} \mmax(k)\) + \mmax(|l|) & l<-1
\end{cases}
\end{align}

\section{Kopplung der gef\"uhrten Moden zweier gerader Faserst\"ucken}
[1989Hill]
Ausgehend von einer Modenverteilung in geraden Faserst\"ucken, kann
die Kopplung durch einen Phasenkeil mit Winkel $\alpha$ vermittelt
werden. Im Paper berechnet der Autor die Ausbreitung in einem hohlen
(luftgefuellten) Wellenleiter und verwendet einen Phasenkeil in
Luft. F\"ur unser Problem findet der wesentliche Teil der Kopplung im
Kern statt und der Phasenkeil wird deshalb auf den Index $\nco$
bezogen.
\begin{align}
  F_{j(l,m)}(r,\phi) &= \psi(r) 
  \begin{cases}
    \cos(l \phi) & l>0\\  
    \sin(l \phi) & l<0
  \end{cases} \\
  \delta L &= R_b \sin\alpha\\
  c_{nm}&=\int\!\!\!\int\! F_n(x,y)F_m(x,y) e^{i\nco k_0\alpha x} \textrm{d}x \textrm{d}y
\end{align}

Alternativ kann man auch bez\"uglich Zylinderkoordinaten
integrieren. Damit l\"a\ss t sich besser feststellen ob ausreichend
Rand mit einbezogen wurde.

\begin{align}
  \Re\{c_{nm}\}&\propto\int_0^{\infty}\!\!\!\!\int_0^{2\pi}\!\!\!\!\!\! \cos(n\phi)\cos(m\phi) \cos(k_0\alpha x) \textrm{d}\phi\, \psi_n(r) \psi_m(r) r\, \textrm{d}r 
\end{align}

Hier bricht $x=r \cos(\phi)$ die Zylindersymmetrie, aber verglichen
mit den oszillierenden Termen aus den Feldern ist die Funktion
$\cos(k_0\alpha x)$ f\"ur kleine Tiltwinkel $\alpha$ langsam
ver\"anderlich. Ausserdem kann man das Produkt
$\cos(l\phi)\cos(m\phi)$ der zwei azimuthalen Feldfunktionen als Summe
von Oszillationen mit der Differenz- und Summenfrequenzen schreiben:

% a:trigsimp(demoivre(expand(exponentialize(cos(l*x)*cos(m*x)))));
% b:trigsimp(demoivre(expand(exponentialize(cos(l*x)*sin(m*x)))));
% c:trigsimp(demoivre(expand(exponentialize(sin(l*x)*sin(m*x)))));

% integrate(cos(l*x)*exp(%i*z*cos(x)),x,0,2*%pi);
% integrate(cos(l*x)*cos(z*cos(x)),x,0,2*%pi);

\begin{align}
  I_\phi(r) &= \int_0^{2\pi}\!\!\!\!\!\! \cos(l\phi)\cos(m\phi) e^{ik_0\alpha\rho r\cos(\phi)} \textrm{d}\phi = \nonumber \\
  &\frac{1}{2}\int_0^{2\pi}\!\!\!\!\!\! \cos\((l+m)\phi\) e^{ik_0\alpha\rho r\cos(\phi)} \textrm{d}\phi  \nonumber \\
  &+ \frac{1}{2}\int_0^{2\pi}\!\!\!\!\!\! \cos\((l-m)\phi\) e^{ik_0\alpha\rho r\cos(\phi)} \textrm{d}\phi \nonumber \\
  &=  \int_0^{\pi}\! e^{ik_0\alpha\rho r\cos(\phi)} \cos\((l+m)\phi\) \textrm{d}\phi  \nonumber \\
  &+ \int_0^{\pi}\! e^{i\underbrace{k_0\alpha\rho r}_z\cos(\phi)} \cos\((l-m)\phi\)  \textrm{d}\phi
\end{align}

Hierbei ist $x=\rho r \cos(\phi)$.

Beide Integrale entsprechen einer Besselfunktion gem\"ass der
Gleichung:
\begin{align}
  J_n(z) = \frac{i^{-n}}{\pi}\int_0^\pi e^{iz\cos\theta} \cos(n \theta) \textrm{d} \theta
\end{align}

\begin{align}
  I_\phi(r) &= \pi \(i^{l+m} J_{l+m}(k_0\alpha\rho r)+i^{l-m} J_{l-m}(k_0\alpha\rho r)\)
\end{align}

Obige Integrale gehen davon aus, dass die Biegung der Faser in
derselben Ebene wie die Modenrichtung erfolgt. Bei einer Verdrehung
der Biegerichtung um den Winkel $\theta$ gegen\"uber dem
Koordinatensystem der Moden ergibt sich hingegen das folgende
Integral:
\begin{align}
  \label{eq:besselint}
  I^{+/\textcolor{red}{-}}_\phi(\theta,r) &= \int_0^{2\pi}
  \begin{pmatrix}
    \cos(\nu\phi)\\
    \textcolor{red}{\sin(\nu\phi)}
  \end{pmatrix}
  e^{ia(\sin(\theta)\cos(\phi)+\cos(\theta)\sin(\phi))} \textrm{d}\phi 
\end{align}


\begin{figure}[!hbt]
  \centering
  \includegraphics{oscillation-arbitrary-bend-angle.pdf}
  \caption{Numerische Integration im azimuthalen Teiles $\Re\(I^+_\phi(\theta,r)\)$ des
    Koppelkoeffizienten gem\"a\ss\ Gleichung \eqref{eq:besselint}
    f\"ur verschiedene Winkel des Biegeradius verglichen mit der
    Orientierung der Modenfelder, $\nu=4$.}
  \label{fig:besselint}
\end{figure}

In \figref{fig:besselint} habe ich den Realteil dieses Integrals
numerisch f\"ur verschiedene Werte von $\theta$ ausgewertet. Offenbar
moduliert dies die H\"ohe der resultierenden Besselfunktion.

\begin{figure}[!hbt]
  \centering
  \includegraphics{oscillation-arbitrary-bend-angle-a5_2.pdf}
  \caption{Variation des Integrals aus \figref{fig:besselint} f\"ur
    $a=5.2$ und $t\in[0,\pi/2]$. Offenbar folgt die Oszillation
    einem Cosinus.}
  \label{fig:besselint2}
\end{figure}

In \figref{fig:besselint2} habe ich das Integral noch einmal
spezifisch an der Stelle $a=5.2$ untersucht. Offenbar variert das
Integral wie $\cos(4\theta)$. D.h. f\"ur $\theta=0$ treten Werte mit
maximalen Betrag auf, w\"ahrend f\"ur $\theta=\pi/8$ das Integral
verschwindet.

F\"ur $\nu=4$ verschwindet der Imagin\"arteil
$\Im\(I^+_\phi(\theta,r)\)$ , wie in \figref{fig:besselint-im} zu sehen
ist.
\begin{figure}[!hbt]
  \centering
  \includegraphics{oscillation-arbitrary-bend-angle-sin.pdf}
  \caption{Numerische Integration im azimuthalen Teiles
    $\Im\(I_\phi(\theta,r)\)$ des Koppelkoeffizienten
    gem\"a\ss\ Gleichung \eqref{eq:besselint} f\"ur verschiedene
    Winkel des Biegeradius verglichen mit der Orientierung der
    Modenfelder, $\nu=4$. Offenbar verschwindet dieses Integral, f\"ur
    diese Darstellung wurden alle Werte mit $10^{12}$ multipliziert.}
  \label{fig:besselint-im}
\end{figure}

F\"ur $\nu=5$ existiert der Imagin\"arteil $\Im\(I^+_\phi(\theta,r)\)$
hingegen. Dies ist in in \figref{fig:besselint-im-nu5} dargestellt.

\begin{figure}[!hbt]
  \centering
  \includegraphics{oscillation-arbitrary-bend-angle-sin-nu5.pdf}
  \caption{Numerische Integration im azimuthalen Teiles $\Im\(I_\phi(\theta,r)\)$ des
    Koppelkoeffizienten gem\"a\ss\ Gleichung \eqref{eq:besselint}
    f\"ur verschiedene Winkel des Biegeradius verglichen mit der
    Orientierung der Modenfelder, $\nu=5$.}
  \label{fig:besselint-im-nu5}
\end{figure}

Die numerischen Experimente geben Anlass f\"ur den folgenden analytischen Ausdruck f\"ur das Integral:
\begin{align}
  \label{eq:besselint}
  I^{+/\textcolor{red}{-}}_\phi(\theta,r) &= \int_0^{2\pi}
  \begin{pmatrix}
    \cos(\nu\phi)\\
    \textcolor{red}{\sin(\nu\phi)}
  \end{pmatrix}
  e^{ia(\sin(\theta)\cos(\phi)+\cos(\theta)\sin(\phi))} \textrm{d}\phi \\
  &=\pi i^\nu \cos(4\theta)  J_\nu(a)
\end{align}

Durch Vergleich mit numerischer Quadratur habe ich folgende
analytische Ausdr\"ucke f\"ur die Integrale bestimmt:
\begin{align}
  \chi &= \sin(\theta)\cos(\phi)+\cos(\theta)\sin(\phi) \\
  \int_0^{2\pi}\cos(l \phi)\cos(r\chi) \textrm{d}\phi &=
  \begin{cases}
    2 \pi \cos(l\theta)  J_l(r) & l\ \textrm{even} \\
    0 & \textrm{else} \end{cases}\\
  \int_0^{2\pi}\sin(l \phi)\cos(r\chi) \textrm{d}\phi &=
  \begin{cases}
    2 J_0(r) & l=0 \\
    -2 \pi \sin(l\theta)  J_l(r) & l\ \textrm{even} \\
    0 & \textrm{else} \end{cases}\\
  \int_0^{2\pi}\cos(l \phi)\sin(r\chi) \textrm{d}\phi &=
  \begin{cases}
    2 \pi \sin(l\theta)  J_l(r) & l\ \textrm{odd} \\
    0 & \textrm{else} \end{cases}\\
  \int_0^{2\pi}\sin(l \phi)\sin(r\chi) \textrm{d}\phi &=
  \begin{cases}
    2 \pi \cos(l\theta)  J_l(r) & l\ \textrm{odd} \\
    0 & \textrm{else} \end{cases}  
\end{align}



Hier haben die offenbar voll abgefahrene Umformungen gemacht, aber im
Paper leider nicht hingeschrieben:
\begin{align}
J_{-l}(x) &= -J_l(x) \\
\sign \nu &= \nu<0 ? -1 : 1 = \nu/|\nu| = |\nu|/\nu \\
J_\nu(x) &= (-1)^{(\sign{\nu}-1)/2} J_{|\nu|}(x) \\ 
& = (-1)^{\frac{1}{2}(\nu/|\nu|-1)} J_{|\nu|}(x) \\ 
& = ((-1)^{\frac{1}{2}})^{\nu/|\nu|-1} J_{|\nu|}(x) \\ 
& = i^{\nu/|\nu|-1} J_{|\nu|}(x)\\
i^{-\mu} J_\mu(x) &= i^{-\mu} i^{\mu/|\mu|-1} J_{|\mu|}(x)\\
&= i^{\mu/|\mu|-1 -\mu } J_{|\mu|}(x) \\
&= i^{\mu/|\mu| -\mu } (-i) J_{|\mu|}(x) \\
&= i^{|\mu|/\mu-\mu} (-i) J_{|\mu|}(x) \\
&= i^{|\mu|/\mu-|\mu| |\mu| /\mu-1} J_{|\mu|}(x) \\
&= i^{|\mu|/\mu-|\mu| |\mu| /\mu-|\mu|/|\mu|} J_{|\mu|}(x)
\end{align}


Mir ist noch unklar, ob man das Integral analytisch f\"ur beliebige
Winkel l\"osen kann.  Bei der numerischen Auswertung des Azimuthalen
Integrals sollte man die Funktion \verb!integration_qawo()! anwenden,
die f\"ur oszillierende Integranden ausgelegt ist und eine Tabelle mit
entsprechenden Chebyshev Momenten benutzt.

Das Integral im Inneren des Kerns kann mit der Funktion
\verb!integration_qag()! ausgewertet werden, die die Gauss-Kronrod
Regeln mit 15 bis 61 Punkten implementiert.  [2012Auluck] diskutiert
die Eigenschaften und eine Approximation des dreifachen Bessel
Integrals, das brauche ich aber hier nicht weiter.
\begin{align}
    c_{nm}^\textrm{in}&=\frac{1}{\sqrt{N_n N_m} J_n(u_n) J_m(u_m)}
    \int_0^1\!\!\!\! I_\phi(r) J_n(u_n r) J_m(u_m r) r\, \textrm{d}r 
\end{align}

Das Integral ausserhalb des Kerns kann mit der Funktion
\verb!integration_qagiu()! ausgewertet werden, die das Interval
$(a,\infty)$ mit der Transformation $r=a+(1-t)/t$ auf $(0,1]$
abbildet.

\begin{align}
    c_{nm}^\textrm{out}&=\frac{1}{\sqrt{N_n N_m}K_n(w_n) K_m(w_m)}
    \int_1^\infty\!\!\!\! I_\phi(r) K_n(w_n r) K_m(w_m r) r\, \textrm{d}r 
\end{align}



\begin{figure}[hbtp]
  \centering
  \includegraphics[width=.7\columnwidth]{bla-coef}
  \includegraphics[width=.7\columnwidth]{bla-coef-phase}
  \caption{Kopplungskonstanten zwischen den 246 Moden f\"ur $v=32$
    oben Absolutbetrag, unten Phase. $\lambda=\unit[500]{nm}$, wedge
    angle $\alpha=\unit[2]{mrad}$.}
  \label{fig:coef}
\end{figure}

\section{Kopplung von Licht an Biegung in Strahlungsmdoden}
An Biegungen kann auch Licht verloren gehen. F\"ur jede der diskreten
Gef\"uhrten Moden kann eine Kopplungskonstante in das kontinierliche
Spektrum der Strahlungsmoden bestimmt werden. Die Strahlungsmoden
haben die Form:
\begin{align}
\E(x,y,z) = (\e_t(x,y)+e_z(x,y)\hz ) \exp(i\beta z)
\end{align}
Die Zeitabh\"angigkeit setzen wir \"uberall $e^{-i\omega t}$. Deshalb
hat die Ausbreitungskonstante $\beta$ von vorw\"artslaufende Moden
einen positiven Realteil. Die transversale Feldkomponente $\e_t$ wird
nicht reell gew\"ahlt.  Uns interessieren hier nicht die evaneszenten
Moden mit $0<\Im\beta<\infty$, diese transportieren schliesslich keine
Energie. In [Snyder] wird der reelle Modenparameter $Q$ eingef\"uhrt,
um evaneszente und sich ausbreitende Strahlungsmoden gleicherma\ss en
behandeln zu k\"onnen.

\begin{align}
 Q = \rho \sqrt{k^2\ncl^2-\beta^2}
\end{align}
Das komplette Strahlungsfeld l\"a\ss t sich damit ausdr\"ucken als:
\begin{align}
\E_\textrm{rad} (x,y,z) = \sum_j \int_0^\infty a_j(Q) \e_j(x,y,Q) e^{i\beta(Q)z} \textrm{d} Q
\end{align}

Uns interessieren nur die ausbreitenden Strahlungsmoden im Bereich
$0<Q\le k\rho\ncl$.

Die Leistung in allen Strahlungsmoden ist:
\begin{align}
P_\textrm{rad}  = \sum_j \int_0^{k\rho\ncl}\!\!\!\!\!\!\!\!\! |a_j(Q)|^2 N_j(Q) \textrm{d}Q
\end{align}
Dieser Ausdruck ist unabh\"angig von $z$ und daher konstant entlang
der Faser.

Die Felder f\"ur Strahlungsmoden der schwach f\"uhrenden
Stufenindexfaser sind:
\begin{align}
\Psi_l^\textrm{even} &= \cos(l\phi) \begin{cases}
p_l J_l(UR)  & 0\le R \le 1 \\
J_l(QR)+q_l H_l^{(1)} (QR) & 1 \le R < \infty
\end{cases}\\
\Psi_l^\textrm{odd} &= \Psi_l^\textrm{even} \tan(l\phi)
\end{align}

\begin{align}
p_l &= \frac{2i}{\pi} (U J_{l+1}(U) H_l^{(1)}(Q)-QJ_l(U)H^{(1)}_{l+1}(Q))^{-1} \\
q_l &= i \frac{\pi}{2} p_l(U J_{l+1}(U) J_l(Q)-QJ_l(U)     J_{l+1}(Q))
\end{align}

Die $\Psi$ werden dann linear kombiniert um $\e_t$ zu bilden. Dann
gibt es einen komplizierten Ausdruck um mittels $\e_t$ $\h_t$, $e_z$
und $h_z$ auszudr\"ucken. Die Normierungskonstante sieht
verh\"aeltnism\"a\ss ig einfach aus. Das Kopplungsintegral ist
vermutlich recht kompliziert numerisch auszuwerten, denn es enth\"alt
die Integration \"uber $\beta$. Au\ss erdem sind die Felder der
Strahlungsmoden nicht rein transversal.


\section{Propagation entlang der Faser}
\begin{align}
E_{j(l,m)}(r,\phi,z) = \psi_m(r)e^{-\beta_j z+ l\phi}
\end{align}

\section{Einkopplung in die Faser}
incident beam is tilted towards fiber axis
and fresnel reflection 
[snyder]
\begin{align}
n_i \sin(\theta_i) &= \nco \sin(\theta_z)\\
\E_t(\theta_z) &= \frac{2 n_i}{\nco+n_i} \E_i(\theta_i)
\end{align}

Schreibe alle Samplewerte des Eingangsfeldes mit minimalen
Absolutbetrag
\begin{align}
  \Omega = \{(x,y)\in\Ainfty: |E_t(x,y)| > \kappa\}
\end{align}
in einen Vektor $w_i = E_t(\Omega_i)$. Konstruiere eine Matrix $Z$
deren Zeilen den entsprechenden Wert der verschiedenen Moden mit
linearem Index $j$ an den Punkten aus der Menge $\Omega$ <enthaelt:
\begin{align}
  Z_{ij} = u_j(\Omega_i) 
\end{align}
Die Normal equation des least square problems:
\begin{align}
(w^T Z) (Z^T Z)^{-1}
\end{align}

Das ist genauer in 15.4.2 numerical recipes 3rd ed beschrieben:
Das lineare Modell sei
\begin{align}
  y(x) = \sum_{k=0}^{M-1} a_k X_k(x)
\end{align}
mit den Basisfunktionen $X_k$. Im Folgenden beschreibt $M$ die Anzahl
der Moden und $N$ die Anzahl an Positionen, an denen eine elektrische
Feldstaerke gemessen wurde. Die Merit Funktion ist
\begin{align}
  \chi^2 = \sum_{i=0}^{N-1}\frac{1}{\sigma_i^2}\(y_i- \sum_{k=0}^{M-1} a_k X_k(x_i\)^2
\end{align}
Die Design Matrix A sei eine Matrix mit $N\times M$ komponenten aus
$M$ Basisfunktionen und $N$ Koordinatenpunkten $x_i$ mit $N$
Messfehlern $\sigma_i$.
\begin{align}
  A_{ij} = X_j(x_i)/\sigma_i
\end{align}
Es sollte gelten $N\ge M$, wir fordern mehr Datenpunkte als
Modelparameter. Weiterhin definieren wir den Datenvektor $\b$ der L\"ange N mit
\begin{align}
  b_i = y_i/\sigma_i
\end{align}
und den Parametervektor $\a$ mit $M$ Eintr\"agen.
\begin{align}
  \a = (a_0, a_1,\dots,a_{M-1})
\end{align}
Um die optimalen Parameter zu bestimmen fordern wir, dass die
Ableitung $\chi^2_{,a_k}$ der Meritfunktion nach allen Parametern
verschwindet. Damit ergeben sich $M$ Gleichungen:
\begin{align}
  0 = \sum_{i=0}^{N-1}\frac{1}{\sigma_i^2}\(y_i- \sum_{j=0}^{M-1} a_j X_j(x_i)\) X_k(x_i) 
\end{align}
mit $k=0,\ldots, M-1$.  Wir vertauschen die Summen um diesen
Ausdruck in Matrixschreibweise bringen:
\begin{align}
  0 &= \sum_{i=0}^{N-1}\(\frac{y_iX_k(x_i)}{\sigma_i^2}- \sum_{j=0}^{M-1} \frac{a_j X_j(x_i)X_k(x_i)}{\sigma_i^2}\)   \\
0 &=\sum_{i=0}^{N-1} \frac{y_iX_k(x_i)}{\sigma_i^2}- \sum_{j=0}^{M-1} \sum_{i=0}^{N-1} \frac{a_j X_j(x_i)X_k(x_i)}{\sigma_i^2}
\end{align}
\begin{align}
\sum_{j=0}^{M-1} \underbrace{\sum_{i=0}^{N-1}  \frac{X_j(x_i)X_k(x_i)}{\sigma_i^2}}_{\alpha_{kj}} a_k =  \underbrace{\sum_{i=0}^{N-1}\frac{y_iX_k(x_i)}{\sigma_i^2}}_{\beta_k} 
\end{align}
\begin{align}
  \sum_{j=0}^{M-1} \alpha_{kj} a_j = \beta_k
\end{align}
Dabei ist $\valpha$ eine $M\times M$ Matrix und $\vbeta$ ein Vektor
der L\"ange $M$. Als Matrixgleichung geschrieben:
\begin{align}
  \valpha \a = \vbeta
\end{align}
oder
\begin{align}
  \valpha &= A^T A\\
  \vbeta &= A^T \b\\
  (A^T A) \a &= A^T \b
\end{align}
Dieses System sollte mit SVD gel\"ost werden. Die Zerlegung einer Matrix $A$ ist 
\begin{align}
  \begin{pmatrix}
  &&&&\\
  &&&&\\ 
  &&A&&\\
  &&&&\\
  &&&&\\ 
  \end{pmatrix}
  =
  \begin{pmatrix}
  &&&&\\
  &&&&\\ 
  &&U&&\\
  &&&&\\
  &&&&\\
  \end{pmatrix}
  \underbrace{
  \begin{pmatrix}
  w_0 \\
  {} & w_1 \\
  {} & {} & \ddots \\
  {} & {} & {} & w_{N-1}
  \end{pmatrix}}_S
  \begin{pmatrix}
  &&\\ 
  & V^T &\\
  &&\\
  \end{pmatrix}
\end{align}
Wir bezeichen mit $U_{(i)}$ die $M$ Spalten von $U$ (Vektoren der
L\"ange $N$) und mit $V_{(i)}$ die $M$ Spalten von $V$ (Vektoren der
L\"ange $M$). Damit ist die L\"osung des linearen
Minimierungsproblems $\min |A\a-\b|^2$ :
\begin{align}
  \a = \sum_{i=0}^{M-1} \(\frac{U_{(i)}\b}{w_i}\)  V_{(i)}
\end{align}
F\"ur kleine Werte von $w_i$ muss der entsprechende Summand Null
gesetzt werden.  Der Fehler der abgesch\"atzten Parameter ergibt sich
als
\begin{align}
  \sigma^2(a_j) = \sum_{i=0}^{M-1} \(V_{ji}/w_i\)^2
\end{align}
Die GNU Scientific Library stellt f\"ur diese Aufgabe die Funktionen \verb!SV_decomp(A,V,S,vec work)! und  \verb!SV_solve(U,V,S,vec b,vec x)! bereit.


\section{Vergleich mit FIMMPROP}
die benutzen auch joints um bends zu repraesentieren

7.4 boundary reflections in the bend

modenkopplung

fimmprop can can easily model loss in single mode fibers (multimode is
*much* harder)!

in manchen faellen koppelt man nur in gleiche p-ordnung, z.b. $TE_{10}$
(morder 1 porder 2) und horizontal gebogene faser

fimmprop kann nur bis ca. 40 moden simulieren

wie geht in meiner simulation energie verloren? ich muesste eigentlich auch noch an strahlungsmoden koppeln

\end{document}

%%% Local Variables: 
%%% mode: latex
%%% TeX-master: t
%%% eval: (reftex-mode)
%%% End: 
